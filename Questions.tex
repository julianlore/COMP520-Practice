\documentclass[12pt]{article}
\usepackage[margin=1in]{geometry}
\usepackage{enumerate, bussproofs, listings, amsmath}
\EnableBpAbbreviations

\makeatletter
\renewcommand{\thesection}{\arabic{section}.}
\renewcommand\section{\@startsection{section}{1}{\z@}%
  {-3.5ex \@plus -1ex \@minus -.2ex}%
                                       {2.3ex \@plus.2ex}%
                                       {\normalfont\bfseries Q}}
\makeatother

\lstset{basicstyle=\ttfamily}

\begin{document}
\section{Scanners and Parsers}
\begin{enumerate}[(a)]
\item \lstinputlisting[breaklines=true]{Heffalump.l}
\item ~ \lstinputlisting[breaklines=true]{Heffalump.y}
\end{enumerate}
\section{Typechecking}
\begin{enumerate}[(a)]
\item ~
  \begin{prooftree}
    \AXC{$F, T, V \vdash isCastable(\tau, \sigma)$}
    \AXC{$T(cast) = \sigma$}
    \AXC{$F, T, V \vdash e: \tau$}
    \TIC{$F, T, V \vdash cast(e) : \sigma$}
  \end{prooftree}
\item Under the context $F, T$ and $V$, if $e_i$ is of type $\tau_i$
  (for $i \in \left\{1, 2, \ldots, n\right\}$) and there is a function
  that takes in parameters $\tau_1, \ldots, \tau_n$ which returns type
  $\tau$, then under the context $F, T$ and $V$, $f(e_1, \ldots, e_n)$
  is of type $\tau$.
\end{enumerate}
\section{Bytecode Generation}
\begin{enumerate}[(a)]
\item \lstinputlisting[breaklines=true]{tree.j}
\item (See above)
\item Expressions must increase the stack height by one. Statements
  must keep the net stack height constant.
\end{enumerate}

\section{Optimization}
\begin{enumerate}[(a)]
\item
  \begin{minipage}{0.5\linewidth}
\begin{verbatim}
iload_{x}               [o]
dup                     [o; o]
aload_{y}               [o; o; v]
swap                    [o; v; o]
putfield {field} {type} [o], size = 0
pop                     []
\end{verbatim}
  \end{minipage}
  $\implies$ \hspace{1 em}
  \begin{minipage}{0.5\linewidth}
\begin{verbatim}
aload_{y}               [o]
iload_{x}               [o; v]
putfield {field} {type} [], size = 0
\end{verbatim}
  \end{minipage}
  The stack is still empty at the end and size still gets assigned
  whatever is at local $0$.

\item $A$ is not sound, \texttt{R3} could change elsewhere before we
  jump to \texttt{\_label}, so it is not equivalent to substituting in \texttt{R1}.
  \\ $B$ is not sound as we no longer store \texttt{R1} into \texttt{R3}.
\end{enumerate}

\section{Native Code Generation}
\begin{enumerate}[(a)]
\item
\begin{tabular}{l l l l}
  name&offset&location&register
  \\
  \\ local & $0$ & & \texttt{R0}
  \\ local & $1$ & & \texttt{R1}
  \\ local & $2$ & \texttt{[fp - 4]}
  \\ stack & $0$ & & \texttt{R2}
  \\ stack & $1$ & & \texttt{R3}
  \\ scratch & $0$ & & \texttt{R4}
  \\ scratch & $1$ & & \texttt{R5}
\end{tabular}
\item
\begin{verbatim}
_baz:
    save sp, -112, sp
    st R0, [fp + 68]
    st R1, [fp + 72]

    mov 0, R2
    st R2, [fp - 4]
    mov R0, R2
    mov R1, R3
    cmp R2, R3

    bge _end
    mov R0, R2
    mov R1, R3
    mul R2, R3, R2
    st R2, [fp - 4]
_end:
    ld [fp - 4] R0
    restore
    ret
\end{verbatim}
\end{enumerate}
\section{Garbage Collection}
\begin{enumerate}[(a)]
\item
  \begin{enumerate}[(i)]
  \item Object 1: 1
    \\ Object 2: 3
  \item Object 1: 0
    \\ Object 2: 2
    \\ Object 3: 2
  \end{enumerate}
\end{enumerate}
\section{Special Topics}
\begin{enumerate}[(a)]
\item
  \begin{enumerate}[(a)]
  \item 4 accesses
  \item 3 accesses
  \end{enumerate}
\item That the type stored into a local is the same as the type of the
  local. No overflow.
\end{enumerate}
\end{document}
